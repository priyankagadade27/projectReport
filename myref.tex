\begin{thebibliography}{111}
\bibliographystyle{IEEEtran}
	\bibitem{lime} Xiaojie Guo,Yu Li, and Haibin Ling,"LIME: Low-Light Image Enhancement via  
	Illumination Map Estimation" IEEE TRANSACTIONS ON IMAGE PROCESSING, VOL. 26, NO. 2, FEBRUARY 2017, 
	pp 982-993. 
	
	\bibitem{retinex} Ankitha A Reddy, Poorvaja R Jois, J Deekshitha, Namratha S, Rajeshwari Hegde 
	"COMPARISON OF IMAGE ENHANCEMENT TECHNIQUES USING RETINEX MODELS" International Journal of Advanced 
	Computer Engineering and Communication Technology (IJACECT),ISSN (Print): 2278-5140, Volume-2, Issue 
	– 3, 2013, pp 7-12

	\bibitem{colorSpace}"http://www.robots.ox.ac.uk/~teo/thesis/Thesis/deCampos3DHandTrackingApp.pdf"
	
	\bibitem{he1} Sos S. Agaian, Blair Silver, and Karen A. Panetta," Transform Coefficient Histogram-
	Based Image Enhancement Algorithms Using Contrast Entropy" IEEE TRANSACTIONS ON IMAGE PROCESSING, 
	VOL. 16, NO. 3, MARCH 2007pp 741-758

	\bibitem{he2} Sayali Nimkar, Sucheta Shrivastava and Sanal Varghese "CONTRAST ENHANCEMENT AND 
	BRIGHTNESS PRESERVATION USING MULTIDECOMPOSITION HISTOGRAM EQUALIZATION" Signal & Image Processing : 
	An International Journal (SIPIJ) Vol.4, No.3, June 2013 pp 83-93
	
	\binitem{amd1} Chirag S. Panchal, Abhay B. Upadhyay, "Depth Estimation Analysis Using Sum of 
	Absolute Difference Algorithm" International Journal of Advanced Research in Electrical, Electronics 
	and Instrumentation Engineering (An ISO 3297: 2007 Certified Organization) Vol. 3, Issue 1, January 
	2014, ISSN (Print) : 2320 – 3765 ISSN (Online): 2278 – 8875 pp 6761-6767
	
	\bibitem{amd2} Hiroaki Niitsuma, Tsutomu Maruyama, "Sum of Absolute Difference Implementations for 
	image processing on FPGAs", 2010 International Conference on Field Programmable Logic and 
	Applications, 978-0-7695-4179-2/10, 2010 IEEE, DOI 10.1109/FPL.2010.40

	\bibitem{amd3} D. V. Manjunatha, Pradeep Kumar and R. Karthik, "FPGA IMPLEMENTATION OF SUM OF 
	ABSOLUTE DIFFERENCE (SAD) FOR VIDEO APPLICATIONS", ARPN Journal of Engineering and Applied Sciences, 
	VOL. 12, NO. 24, DECEMBER 2017, ISSN 1819-6608, pp 7192-7197

	\bibitem{entropy1} Liyun Zhuang and Yepeng Guan, "Research Article-Adaptive Image Enhancement Using 
	Entropy-Based Subhistogram Equalization", Hindawi, Computational Intelligence and 
	Neuroscience,Volume 2018, Article ID 3837275, 13 pages,https://doi.org/10.1155/2018/3837275

	\bibitem{entropy2} Gilson A. Giraldi,Paulo S.S. Rodrigues,  "Principle of Maximum Entropy for 
	Histogram Transformation and Image Enhancement", National Laboratory for Scientific Computing Petr
	´opolis, RJ, Brazil and Computer Science and Electrical Engineering Departments of the FEI 
	Technologic University S˜ao Bernardo do Campo, SP, Brazil
	
	\bibitem{plt1} Anita G. Khandizod, R.R. Deshmukh, "Comparative Analysis of Image Enhancement 
	Technique for Hyperspectral Palmprint Images", International Journal of Computer Applications (0975 
	– 8887)Volume 121 – No.23, July 2015, pp 30-35.
	
	\bibitem{rmse1} Krishan Kant Lavania, Shivali, Rajiv Kumar, "Image Enhancement using Filtering 
	Techniques", Krishan Kant Lavania et al. / International Journal on Computer Science and Engineering 
	(IJCSE), Vol. 4 No. 01 January 2012, ISSN : 0975-3397, pp 14-20.
	
	\bibitem{dip1} R.C. GONZÁLEZ, R. E. W. Digital Image Processing. Pearson Prentice Hall.:Pearson 
	Prentice Hall, 2008. Texto básico y completo, con muchos ejemplos., 2008.

	\bibitem{dip2} W.O. SAXTON, T.J. PITT and M. HORNER, "DIGITAL IMAGE PROCESSING: THE SEMPER SYSTEM ", 
	Ultramicroscopy 4 (1979) 343-354 © North-Holland Publishing Company

	\bibitem{dip3} Qieshi Zhang, Sei-ichiro Kamata,"A NOVEL COLOR SPACE BASED ON RGB COLOR BARYCENTER", 
	978-1-4799-9988-0/16/ ©2016 IEEE, ICASSP 2016 pp 1601-1605

	\bibitem{dip4} Oleksii Sidorov, Gjovik, Norway, "Novel Approach to Uniformization of a Color Space
	via Generic Deep Learning-Based Transformation", 978-1-5386-5645-7/18/ ©2018 IEEE.

	\bibitem{ie1}H. Cheng and X. Shi, “A simple and effective histogram equalization approach to image 
	enhancement,” Digit. Signal Process., vol. 14, no. 2,pp. 158–170, 2004
	
	\bibitem{he3} H. Cheng and X. Shi, “A simple and effective histogram equalization approach to image 
	enhancement,” Digit. Signal Process., vol. 14, no. 2, pp. 158–170, 2004
	
	\bibitem{ie2} C. Lee, C. Lee, and C.-S. Kim, “Contrast enhancement based on layered difference 
	representation of 2D histograms,” IEEE Trans. Image Process., vol. 22, no. 12, pp. 5372–5384, Dec. 
	2013.

	\bibitem{ill1} S. Wang, J. Zheng, H.-M. Hu, and B. Li, “Naturalness preserved enhancement algorithm 
	for non-uniform illumination images,” IEEE Trans. Image Process., vol. 22, no. 9, pp. 3538–3578, 
	Sep. 2013
	
	\bibitem{ill2} X. Fu, D. Zeng, Y. Huang, X. Zhang, and X. Ding, “A weighted variational model for 
	simultaneous reflectance and illumination estimation,” in Proc. CVPR, 2016, pp. 2782–2790.

	\bibitem{lime2} L. Li, R. Wang, W. Wang, and W. Gao, “A low-light image enhancement method for both 
	denoising and contrast enlarging,” in Proc. ICIP, 2015, pp. 3730–3734.

	\bibitem{colorSpace2} J. T. Barron and J. Malik, “Color constancy, intrinsic images, and shape 
	estimation,” in Proc. ECCV, 2012, pp. 57–70.	

	\bibitem{colorSpace3} B. Funt and L. Shi, “The rehabilitation of MaxRGB,” in Proc. Color Imag. 
	Conf., 2010, pp. 256–259.

	\bibitem{ill3} H. R. Z. Joze, M. S. Drew, G. D. Finlayson, and P. A. T. Rey, “The role of bright 
	pixels in illumination estimation,” in Proc. Eur. Conf. Colour Graph., Imag. Vis., 2012, pp. 41–46.

	\bibitem{lime3} Y. Li, F. Guo, R. Tan, and M. Brown, “A contrast enhancement framework with JPEG 
	artifacts suppression,” in Proc. ECCV, 2014, pp. 174–188.

	\bibitem{colorSpace4} J. Vazquz-Corral, M. Vanrell, R. Baldrich, and F. Tous, “Color constancy by 
	category correlation,” IEEE Trans. Image Process., vol. 21, no. 4,pp. 1997–2007, Apr. 2012.

	\bibitem{plt2} Anita G. Khandizod, R.R. Deshmukh, "Comparative Analysis of Image Enhancement 
	Technique for Hyperspectral Palmprint Images", International Journal of Computer Applications (0975 
	– 8887) Volume 121 – No.23, July 2015, pp 30-35

	\bibitem{plt3}T.Romen Singh, O.Imocha Singh , Kh. Manglem Singh , Tejmani Sinam and Th. Rupachandra 
	Singh, "Image Enhancement by Adaptive Power-Law Transformations", Bahria University Journal of 
	Information & Communication Technology Vol. 3, Issue 1, December 2010, pp 29-37

	\bibitem{plt4} Raman Maini and Himanshu Aggarwal, "A Comprehensive Review of Image Enhancement 
	Techniques", JOURNAL OF COMPUTING, VOLUME 2, ISSUE 3, MARCH 2010, ISSN 2151-9617, pp 8-13.

	\bibitem{rmse2} Roopali D Pai, Prof. Srinivas Halvi, Prof. Basavaraj Hiremath, "Medical Color Image 
	Enhancement using Wavelet Transform and Contrast Stretching Technique", nternational Journal of 
	Scientific and Research Publications, Volume 5, Issue 7, July 2015 1 ISSN 2250-3153, pp 1-7.

	\bibitem{ssr} Doo Hyun Choi, Ick Hoon Jang, Mi Hye Kim, and Nam Chul Kim, "COLOR IMAGE ENHANCEMENT 
	USING SINGLE-SCALE RETINEX BASED ON AN IMPROVED IMAGE FORMATION MODEL" 16th European Signal 
	Processing Conference (EUSIPCO 2008), Lausanne, Switzerland, August 25-29, 2008, copyright by 
	EURASIP
	
	\bibitem{msr} Daniel J. Jobson, Zia-ur Rahman, and Glenn A. Woodell "A Multiscale Retinex for 
	Bridging the Gap Between Color Images and the Human Observation of Scenes" IEEE TRANSACTIONS ON 
	IMAGE PROCESSING, VOL. 6, NO. 7, JULY 1997, pp 965-976.

	\bibitem{msrcr} Zia-ur Rahman,Daniel J. Jobson and Glenn A. Woodell, "MULTI-SCALE RETINEX FOR COLOR 
	IMAGE ENHANCEMENT" , 0-7803-3258-X/96/ 1996 IEEE, pp 1003-1006.

	\bibitem{msrcr2} Hanumantharaju M. C, Ravishankar .M, Rameshbabu D. R, Ramachandran .S, "Color Image 
	Enhancement using Multiscale Retinex with Modified Color Restoration Technique" 2011 Second
	International Conference on Emerging Applications of Information Technology

	

%\bibitem{Yuh00} Yuh J.,(2000). ``{Design and Control of Autonomous Underwater Robots: A Survey}'' \emph{Autonomous Robots} {\em{8\,}},07--24.



\end{thebibliography}