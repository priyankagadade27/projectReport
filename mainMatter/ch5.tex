\chapter{Result and Discussion}
In this project, many low light images tested but only three images result kept in this report along with original images and enhanced images using different Retinex model, Power Law Transformation, Histogram Equalization and Adaptive Histogram Equalization. Also compared all these results with respect to Absolute Mean Difference(AMD),Root Mean Square and Entropy
\section{Retinex Model}
The main focus of this work is to enhance the low light images and this project is implemented in MATLAB tool. Figure \ref{fig:ssr} shows the original low light image of lamp and enhanced image using Single-Scale Retinex(SSR) model. It is clear enhancement in Figure \ref{fig:ssr}(b). The background objects of in figure \ref{fig:ssr}(b) are more clear than original image. The Absolute Mean Difference(AMD),Root Mean Square and Entropy for Single Scale Retinex(SSR) Model of enhanced image of lamp is $69.8401,15.7384$ and $4.9043$ respectively      



\begin{figure}
	\begin{subfigure}{8cm}
		\centering    
    	\includegraphics[width=7cm,height=9cm,keepaspectratio]{images/ch5/bulb_input.jpg}
    	\caption{} 
    \end{subfigure}
  	\begin{subfigure}{6cm}
  		\centering
  		\includegraphics[width=7cm,height=9cm,keepaspectratio]{images/ch5/bulb_ssr.jpg}
   		\caption{}
  	\end{subfigure}
  	\caption{a) Input Image b)Enhanced Image using SSR}
  	\label{fig:ssr}
\end{figure}

Figure \ref{fig:msr} shows the original low light image of lamp and enhanced image using Multi-Scale Retinex(MSR) model. It is clear enhancement in Figure \ref{fig:msr}(b). The background objects of in figure \ref{fig:msr}(b) are more clear than original image. The Absolute Mean Difference(AMD),Root Mean Square and Entropy for Multi-Scale Retinex(MSR) model of enhanced image of lamp is $69.9617, 15.7384$ and $5.0033$ respectively      


\begin{figure}
	\begin{subfigure}{8cm}
		\centering    
    	\includegraphics[width=7cm,height=9cm,keepaspectratio]{images/ch5/bulb_input.jpg}
    	\caption{} 
    \end{subfigure}
  	\begin{subfigure}{6cm}
  		\centering
  		\includegraphics[width=7cm,height=9cm,keepaspectratio]{images/ch5/bulb_msr.jpg}
   		\caption{}
  	\end{subfigure}
  	\caption{a) Input Image b)Enhanced Image using MSR}
  	\label{fig:msr}
\end{figure}

Figure \ref{fig:msrcr} shows the original low light image of lamp and enhanced image using Multi-Scale Retinex with Color Restoration(MSRCR) . It is clear enhancement in Figure \ref{fig:msrcr}(b). The background objects of in figure \ref{fig:msrcr}(b) are more clear than original image. The Absolute Mean Difference(AMD),Root Mean Square and Entropy for Multi-Scale Retinex with Color Restoration((MSRCR) model of enhanced image of lamp is $67.0909,15.7330$ and $6.3657$ respectively      

\begin{figure}
	\begin{subfigure}{8cm}
		\centering    
    	\includegraphics[width=7cm,height=9cm,keepaspectratio]{images/ch5/bulb_input.jpg}
    	\caption{} 
    \end{subfigure}
  	\begin{subfigure}{6cm}
  		\centering
  		\includegraphics[width=7cm,height=9cm,keepaspectratio]{images/ch5/bulb_msrcr.jpg}
   		\caption{}
  	\end{subfigure}
  	\caption{a) Input Image b)Enhanced Image using MSRCR}
  	\label{fig:msrcr}
\end{figure}


Now consider second input image of palace. This image is also taken in low light and will test it for different Retinex Models. Figure \ref{fig:ssrPalace} shows the original low light image of palace and enhanced image using Single-Scale Retinex(SSR) model. In the original image (figure \ref{fig:ssrPalace} (a))  the palace,the valley, edges of rock,small bushy plants and lamps are not clear but these are very clear in the enhanced image(Figur \ref{fig:ssrPalace} (b)). The Absolute Mean Difference(AMD),Root Mean Square and Entropy for Single Scale Retinex(SSR) Model of enhanced image of palace is $103.9633,15.9031$ and $6.8944$ respectively      

\begin{figure}
	\begin{subfigure}{8cm}
		\centering    
    	\includegraphics[width=7cm,height=9cm,keepaspectratio]{images/ch5/palace_input.jpg}
    	\caption{} 
    \end{subfigure}
  	\begin{subfigure}{6cm}
  		\centering
  		\includegraphics[width=7cm,height=9cm,keepaspectratio]{images/ch5/palace_ssr.jpg}
   		\caption{}
  	\end{subfigure}
  	\caption{a) Input Image b)Enhanced Image using SSR}
  	\label{fig:ssrPalace}
\end{figure}


%AMD_SSR =  103.9633
%r_SSR =  15.9031
%en_SSR =   6.8944
%AMD_MSR =  104.7167
%r_MSR =   15.9115
%en_MSR =    7.1499
%AMD_MSRCR =   92.2710
%r_MSRCR =   15.9138
%en_MSRCR =    7.7566

Figure \ref{fig:msrPalace} shows the original low light image of palace and enhanced image using Multi-Scale Retinex(MSR) model. In the original image (figure \ref{fig:msrPalace} (a))  the palace,the valley, edges of rock,small bushy plants and lamps are not clear but these are very clear in the enhanced image(Figure \ref{fig:msrPalace} (b)). The Absolute Mean Difference(AMD),Root Mean Square and Entropy for Multi Scale Retinex(MSR) Model of enhanced image of palace is $104.7167, 15.9115$ and $7.1499$ respectively      

\begin{figure}
	\begin{subfigure}{8cm}
		\centering    
    	\includegraphics[width=7cm,height=9cm,keepaspectratio]{images/ch5/palace_input.jpg}
    	\caption{} 
    \end{subfigure}
  	\begin{subfigure}{6cm}
  		\centering
  		\includegraphics[width=7cm,height=9cm,keepaspectratio]{images/ch5/palace_msr.jpg}
   		\caption{}
  	\end{subfigure}
  	\caption{a) Input Image b)Enhanced Image using MSR}
  	\label{fig:msrPalace}
\end{figure}

Figure \ref{fig:msrcrPalace} shows the original low light image of palace and enhanced image using Multi-Scale Retinex with COlor Restoration(MSRCR) model. In the original image (figure \ref{fig:msrcrPalace} (a))  the palace,the valley, edges of rock,small bushy plants and lamps are not clear but these are very clear in the enhanced image(Figure \ref{fig:msrcrPalace} (b)). The Absolute Mean Difference(AMD),Root Mean Square and Entropy for Multi Scale Retinex with COlor Restoration(MSRCR) Model of enhanced image of palace is $92.2710, 15.9138$ and $7.7566$ respectively      

\begin{figure}
	\begin{subfigure}{8cm}
		\centering    
    	\includegraphics[width=7cm,height=9cm,keepaspectratio]{images/ch5/palace_input.jpg}
    	\caption{} 
    \end{subfigure}
  	\begin{subfigure}{6cm}
  		\centering
  		\includegraphics[width=7cm,height=9cm,keepaspectratio]{images/ch5/palace_msrcr.jpg}
   		\caption{}
  	\end{subfigure}
  	\caption{a) Input Image of Palace b)Enhanced Image of Palace using MSRCR}
  	\label{fig:msrcrPalace}
\end{figure}


Now consider third input image of robot. This image is also taken in low light and will test it for different Retinex Models. Figure \ref{fig:ssrRobot} shows the original low light image of robot and enhanced image using Single-Scale Retinex(SSR) model. In the original image of robot (figure \ref{fig:ssrRobot} (a))  the robot,laptop, laptop keypad and background strips are not clear but these are very clear in the enhanced image(Figur \ref{fig:ssrRobot} (b)). The Absolute Mean Difference(AMD),Root Mean Square and Entropy for Single Scale Retinex(SSR) Model of enhanced image of robot are $76.3404, 15.9245$ and $6.8248$ respectively      


\begin{figure}
	\begin{subfigure}{8cm}
		\centering    
    	\includegraphics[width=7cm,height=9cm,keepaspectratio]{images/ch5/robot_input.jpg}
    	\caption{} 
    \end{subfigure}
  	\begin{subfigure}{6cm}
  		\centering
  		\includegraphics[width=7cm,height=9cm,keepaspectratio]{images/ch5/robot_ssr.jpg}
   		\caption{}
  	\end{subfigure}
  	\caption{a) Input Image of Robot b)Enhanced Image of Robot using SSR}
  	\label{fig:ssrRobot}
\end{figure}



%AMD_SSR =   76.3404
%r_SSR =   15.9245
%en_SSR =    6.8248
%AMD_MSR =   76.3157
%r_MSR =   15.9242
%en_MSR =    6.9686
%AMD_MSRCR =   74.7684
%r_MSRCR =   15.9179
%en_MSRCR =    7.0533
Figure \ref{fig:msrRobot} shows the original low light image of robot and enhanced image using Multi-Scale Retinex(MSR) model. In the original image of robot (figure \ref{fig:msrRobot} (a))  the robot,laptop, laptop keypad and background strips are not clear but these are very clear in the enhanced image(Figur \ref{fig:msrRobot} (b)). The Absolute Mean Difference(AMD),Root Mean Square and Entropy for Multi Scale Retinex(MSR) Model of enhanced image of robot are $76.3157, 15.9242$ and $6.9686$ respectively      


\begin{figure}
	\begin{subfigure}{8cm}
		\centering    
    	\includegraphics[width=7cm,height=9cm,keepaspectratio]{images/ch5/robot_input.jpg}
    	\caption{} 
    \end{subfigure}
  	\begin{subfigure}{6cm}
  		\centering
  		\includegraphics[width=7cm,height=9cm,keepaspectratio]{images/ch5/robot_msr.jpg}
   		\caption{}
  	\end{subfigure}
  	\caption{a) Input Image of Robot b)Enhanced Image of Robot using MSR}
  	\label{fig:msrRobot}
\end{figure}

Figure \ref{fig:msrcrRobot} shows the original low light image of robot and enhanced image using Multi-Scale Retinex with Color Restoration(MSRCR) model. In the original image of robot (figure \ref{fig:msrcrRobot} (a))  the robot,laptop, laptop keypad and background strips are not clear but these are very clear in the enhanced image(Figur \ref{fig:msrcrRobot} (b)). The Absolute Mean Difference(AMD),Root Mean Square and Entropy for Multi Scale Retinex with Color Restoration(MSRCR) Model of enhanced image of robot are $74.7684, 15.9179$ and $7.0533$ respectively      

\begin{figure}
	\begin{subfigure}{8cm}
		\centering    
    	\includegraphics[width=7cm,height=9cm,keepaspectratio]{images/ch5/robot_input.jpg}
    	\caption{} 
    \end{subfigure}
  	\begin{subfigure}{6cm}
  		\centering
  		\includegraphics[width=7cm,height=9cm,keepaspectratio]{images/ch5/robot_msrcr.jpg}
   		\caption{}
  	\end{subfigure}
  	\caption{a) Input Image of Robot b)Enhanced Image of Robot using MSRCR}
  	\label{fig:msrcrRobot}
\end{figure}


\section{Power Law Transformation}
Now test above three images like bulb, palace and robot for Power Law Transformation. Figure \ref{fig:powerLaw} shows the original low light image of lamp and enhanced image using Power Law Transformation . It is clear enhancement in Figure \ref{fig:powerLaw}(b). The background objects of in figure \ref{fig:powerLaw}(b) are more clear than original image. The Absolute Mean Difference(AMD),Root Mean Square and Entropy for Power Law Transformation of enhanced image of lamp is $0.1068,0.1233$ and $5.2476$ respectively      


\begin{figure}
	\begin{subfigure}{8cm}
		\centering    
    	\includegraphics[width=7cm,height=9cm,keepaspectratio]{images/ch5/bulb_input.jpg}
    	\caption{} 
    \end{subfigure}
  	\begin{subfigure}{6cm}
  		\centering
  		\includegraphics[width=7cm,height=9cm,keepaspectratio]{images/ch5/bulb_power.jpg}
   		\caption{}
  	\end{subfigure}
  	\caption{a) Input Image b)Enhanced Image using Power Law Transformation}
  	\label{fig:powerLaw}
\end{figure}

Figure \ref{fig:palacePowerLaw} shows the original low light image of palace and enhanced image using Power Law Transformation. In the original image of palace (figure \ref{fig:palacePowerLaw} (a))  the palace,the valley, edges of rock,small bushy plants and lamps are not clear but these are very clear in the enhanced image(Figure \ref{fig:palacePowerLaw} (b)). The Absolute Mean Difference(AMD),Root Mean Square and Entropy for Power Law Transformation of enhanced image of palace is $0.1857, 0.1957$ and $7.2592$ respectively      

%AMD_power =0.1857
%r_power =0.1957
%en_power =7.2592
\begin{figure}
	\begin{subfigure}{8cm}
		\centering    
    	\includegraphics[width=7cm,height=9cm,keepaspectratio]{images/ch5/palace_input.jpg}
    	\caption{} 
    \end{subfigure}
  	\begin{subfigure}{6cm}
  		\centering
  		\includegraphics[width=7cm,height=9cm,keepaspectratio]{images/ch5/palace_power.jpg}
   		\caption{}
  	\end{subfigure}
  	\caption{a) Input Image of Palace b)Enhanced Image of palace using Power Law Transformation}
  	\label{fig:palacePowerLaw}
\end{figure}

Now consider third input image of robot. This image is also taken in low light and will test it for Power Law Transformation Figure \ref{fig:robotPowerLaw} shows the original low light image of robot and enhanced image using  Power Law Transformation. In the original image of robot (figure \ref{fig:robotPowerLaw} (a))  the robot,laptop, laptop keypad and background strips are not clear but these are very clear in the enhanced image(Figur \ref{fig:robotPowerLaw} (b)). The Absolute Mean Difference(AMD),Root Mean Square and Entropy for  Power Law Transformation of enhanced image of robot are $0.1953, 0.2010$ and $6.2899$ respectively      

%AMD_power =0.1953
%r_power =0.2010
%en_power =6.2899

\begin{figure}
	\begin{subfigure}{8cm}
		\centering    
    	\includegraphics[width=7cm,height=9cm,keepaspectratio]{images/ch5/robot_input.jpg}
    	\caption{} 
    \end{subfigure}
  	\begin{subfigure}{6cm}
  		\centering
  		\includegraphics[width=7cm,height=9cm,keepaspectratio]{images/ch5/robot_power.jpg}
   		\caption{}
  	\end{subfigure}
  	\caption{a) Input Image of Robot b)Enhanced Image of Robot using Power Law Transformation}
  	\label{fig:robotPowerLaw}
\end{figure}


\section{Histogram Equalization}
Now test above same three images like bulb, palace and robot for Power Law Transformation. Figure \ref{fig:histEqu} shows the original low light image of lamp and enhanced image using Histogram Equalization. It is clear enhancement in Figure \ref{fig:histEqu}(b). The background objects of in figure \ref{fig:histEqu}(b) are more clear than original image. The Absolute Mean Difference(AMD),Root Mean Square and Entropy for Histogram Equalization of enhanced image of lamp is $104.7539,15.7363$ and $4.6140$ respectively      


\begin{figure}
	\begin{subfigure}{8cm}
		\centering    
    	\includegraphics[width=7cm,height=9cm,keepaspectratio]{images/ch5/bulb_input.jpg}
    	\caption{} 
    \end{subfigure}
  	\begin{subfigure}{6cm}
  		\centering
  		\includegraphics[width=7cm,height=9cm,keepaspectratio]{images/ch5/bulb_hist_equ.jpg}
   		\caption{}
  	\end{subfigure}
  	\caption{a) Input Image b)Enhanced Image using Histogram Equalization}
  	\label{fig:histEqu}
\end{figure}

Figure \ref{fig:palaceHistEq} shows the original low light image of palace and enhanced image using Histogram Equalization. In the original image of palace (figure \ref{fig:palaceHistEq} (a))  the palace,the valley, edges of rock,small bushy plants and lamps are not clear but these are very clear in the enhanced image(Figure \ref{fig:palaceHistEq} (b)). The Absolute Mean Difference(AMD),Root Mean Square and Entropy for Histogram Equalization of enhanced image of palace are $89.3506, 15.6912$ and $5.9325$ respectively      

%AMD_hist =89.3506
%r_hist =15.6912
%en_hist =5.9325

\begin{figure}
	\begin{subfigure}{8cm}
		\centering    
    	\includegraphics[width=7cm,height=9cm,keepaspectratio]{images/ch5/palace_input.jpg}
    	\caption{} 
    \end{subfigure}
  	\begin{subfigure}{6cm}
  		\centering
  		\includegraphics[width=7cm,height=9cm,keepaspectratio]{images/ch5/palace_hist_equ.jpg}
   		\caption{}
  	\end{subfigure}
  	\caption{a) Input Image of Palace b)Enhanced Image of palace using Histogram Equalization}
  	\label{fig:palaceHistEq}
\end{figure}

Now consider third input image of robot. This image is also taken in low light and will test it for Histogram Equalization Figure \ref{fig:robotHistEq} shows the original low light image of robot and enhanced image using  Histogram Equalization. In the original image of robot (figure \ref{fig:robotHistEq} (a))  the robot,laptop, laptop keypad and background strips are not clear but these are very clear in the enhanced image(Figur \ref{fig:robotHistEq} (b)). The Absolute Mean Difference(AMD),Root Mean Square and Entropy for  Histogram Equalization of enhanced image of robot are $93.1594, 15.1385$ and $5.5086$ respectively      

%AMD_hist =   93.1594
%r_hist =   15.1385
%en_hist =5.5086


\begin{figure}
	\begin{subfigure}{8cm}
		\centering    
    	\includegraphics[width=7cm,height=9cm,keepaspectratio]{images/ch5/robot_input.jpg}
    	\caption{} 
    \end{subfigure}
  	\begin{subfigure}{6cm}
  		\centering
  		\includegraphics[width=7cm,height=9cm,keepaspectratio]{images/ch5/robot_hist_equ.jpg}
   		\caption{}
  	\end{subfigure}
  	\caption{a) Input Image of Robot b)Enhanced Image of Robot using Histogram Equalization}
  	\label{fig:robotHistEq}
\end{figure}



\section{Adaptive Histogram Equalization}
Now test same above three images like bulb, palace and robot for Power Law Transformation. Figure \ref{fig:AHE} shows the original low light image of lamp and enhanced image using Adaptive Histogram Equalization. It is clear enhancement in Figure \ref{fig:AHE}(b). The background objects of in figure \ref{fig:AHE}(b) are more clear than original image. The Absolute Mean Difference(AMD),Root Mean Square and Entropy for Adaptive Histogram Equalization of enhanced image of lamp is $12.9948,10.1784$ and $5.6053$ respectively      


\begin{figure}
	\begin{subfigure}{8cm}
		\centering    
    	\includegraphics[width=7cm,height=9cm,keepaspectratio]{images/ch5/bulb_input.jpg}
    	\caption{} 
    \end{subfigure}
  	\begin{subfigure}{6cm}
  		\centering
  		\includegraphics[width=7cm,height=9cm,keepaspectratio]{images/ch5/bulb_adapt_hist.jpg}
   		\caption{}
  	\end{subfigure}
  	\caption{a) Input Image b)Enhanced Image using Adaptive Histogram Equalization}
  	\label{fig:AHE}
\end{figure}


Figure \ref{fig:palaceAHE} shows the original low light image of palace and enhanced image using Adaptive Histogram Equalization. In the original image of palace (figure \ref{fig:palaceAHE} (a))  the palace,the valley, edges of rock,small bushy plants and lamps are not clear but these are very clear in the enhanced image(Figure \ref{fig:palaceAHE} (b)). The Absolute Mean Difference(AMD),Root Mean Square and Entropy for Adaptive Histogram Equalization of enhanced image of palace are $40.3042, 14.5370$ and $7.4680$ respectively      

%AMD_adaptive =   40.3042
%r_adaptive =   14.5370
%en_adaptive =7.4680

\begin{figure}
	\begin{subfigure}{8cm}
		\centering    
    	\includegraphics[width=7cm,height=9cm,keepaspectratio]{images/ch5/palace_input.jpg}
    	\caption{} 
    \end{subfigure}
  	\begin{subfigure}{6cm}
  		\centering
  		\includegraphics[width=7cm,height=9cm,keepaspectratio]{images/ch5/palace_adapt_hist.jpg}
   		\caption{}
  	\end{subfigure}
  	\caption{a) Input Image of Palace b)Enhanced Image of palace using Adaptive Histogram Equalization}
  	\label{fig:palaceAHE}
\end{figure}

Now consider third input image of robot. This image is also taken in low light and will test it for Adaptive Histogram Equalization Figure \ref{fig:robotAHE} shows the original low light image of robot and enhanced image using Adaptive Histogram Equalization. In the original image of robot (figure \ref{fig:robotAdaptive} (a))  the robot,laptop, laptop keypad and background strips are not clear but these are very clear in the enhanced image(Figure \ref{fig:robotAdaptive} (b)). The Absolute Mean Difference(AMD),Root Mean Square and Entropy for  Adaptive Histogram Equalization of enhanced image of robot are $30.0744, 12.6623$ and $7.2511$ respectively      

%AMD_adaptive =30.0744
%r_adaptive =12.6623
%en_adaptive =    7.2511


\begin{figure}
	\begin{subfigure}{8cm}
		\centering    
    	\includegraphics[width=7cm,height=9cm,keepaspectratio]{images/ch5/robot_input.jpg}
    	\caption{} 
    \end{subfigure}
  	\begin{subfigure}{6cm}
  		\centering
  		\includegraphics[width=7cm,height=9cm,keepaspectratio]{images/ch5/robot_adapt_hist.jpg}
   		\caption{}
  	\end{subfigure}
  	\caption{a) Input Image of Robot b)Enhanced Image of Robot using Adaptive Histogram Equalization}
  	\label{fig:robotAHE}
\end{figure}

Now compare these three images using different Retinex Models, Power Law Transformation, Histogram Equalization and Adaptive Histogram Equalization shown in table \ref{tab:lampTable},\ref{tab:palaceTable} and \ref{tab:robotTable} .

\begin{table}
	%\begin{center}
	\caption{Comparison for Lamp Image.}
    \label{tab:lampTable}
    \center
    \begin{tabular}{ |p{2cm}||p{2cm}|p{2cm}|p{2cm}|  }

 		\hline
	 	%\multirow{3}{*}{\rotatebox[origin=c]{90}{rota}
 		\multirow{2}{*}{\rotatebox[origin=c]{0}{Technique}}&\multicolumn{3}{|c|}{Lamp} \\
 		\cline{2-4}
  			&AMD&RMSE&Entropy    \\
 		\hline
 			SSR& 69.840  &15.7384 &4.9043 \\
 		\hline
 			MSR &69.9617 &15.7384 &5.0033  \\
 		\hline
 			MSRCR& 67.0909  &15.7330 &6.3657 \\
 		\hline
 			PLT& 0.1068  &0.1233 &5.2476\\
 		\hline
 			HE& 104.7539  &15.7363 &4.6140 \\
 		\hline
 			AHE& 12.9948  & 10.1784& 5.6053 \\
 		\hline
	\end{tabular}
\end{table}

%SSR\\
%AMD = 69.8401
%r = 15.7384
%en = 4.9043
%MSR\\
%AMD = 69.9617
%r = 15.7384
%en = 5.0033
%MSRCR\\
%AMD = 67.0909
%r = 15.7330
%en = 6.3657    
    
%Power Transformation\\
%AMD = 0.1068
%r = 0.1233
%en = 5.2476

%Histogram Equalization\\
%AMD = 104.7539
%r = 15.7363
%en = 4.6140

%Adaptive Histogram Equalization\\
%AMD = 12.9948
%r = 10.1784
%en = 5.6053


\begin{table}
	%\begin{center}
	\caption{Comparison for Palace Image.}
    \label{tab:palaceTable}
    \center
    \begin{tabular}{ |p{2cm}||p{2cm}|p{2cm}|p{2cm}|  }

 		\hline
	 	%\multirow{3}{*}{\rotatebox[origin=c]{90}{rota}
 		\multirow{2}{*}{\rotatebox[origin=c]{0}{Technique}}&\multicolumn{3}{|c|}{Lamp} \\
 		\cline{2-4}
  			&AMD&RMSE&Entropy    \\
 		\hline
 			SSR& 69.840  &15.7384 &4.9043 \\
 		\hline
 			MSR &69.9617 &15.7384 &5.0033  \\
 		\hline
 			MSRCR& 67.0909  &15.7330 &6.3657 \\
 		\hline
 			PLT& 0.1068  &0.1233 &5.2476\\
 		\hline
 			HE& 104.7539  &15.7363 &4.6140 \\
 		\hline
 			AHE& 12.9948  & 10.1784& 5.6053 \\
 		\hline
	\end{tabular}
\end{table}


\begin{table}
	%\begin{center}
	\caption{Comparison for Lamp Image.}
    \label{tab:robotTable}
    \center
    \begin{tabular}{ |p{2cm}||p{2cm}|p{2cm}|p{2cm}|  }

 		\hline
	 	%\multirow{3}{*}{\rotatebox[origin=c]{90}{rota}
 		\multirow{2}{*}{\rotatebox[origin=c]{0}{Technique}}&\multicolumn{3}{|c|}{Lamp} \\
 		\cline{2-4}
  			&AMD&RMSE&Entropy    \\
 		\hline
 			SSR& 69.840  &15.7384 &4.9043 \\
 		\hline
 			MSR &69.9617 &15.7384 &5.0033  \\
 		\hline
 			MSRCR& 67.0909  &15.7330 &6.3657 \\
 		\hline
 			PLT& 0.1068  &0.1233 &5.2476\\
 		\hline
 			HE& 104.7539  &15.7363 &4.6140 \\
 		\hline
 			AHE& 12.9948  & 10.1784& 5.6053 \\
 		\hline
	\end{tabular}
\end{table}
