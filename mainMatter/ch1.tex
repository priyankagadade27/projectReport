\chapter{Introduction}
When one captures images in low-light conditions.the images often suffer from low visibility. Besides degrading the visual aesthetics of images, this poor quality may also significantly degenerate the performance of many computer vision and multimedia algorithms that are primarily designed for high-quality inputs. In this paper, we propose a simple yet effective lowlight image enhancement (LIME) method. More concretely, the illumination of each pixel is first estimated individually by finding
the maximum value in R, G, and B channels. Furthermore,we refine the initial illumination map by imposing a structure prior on it, as the final illumination map. Having the wellconstructed illumination map, the enhancement can be achieved accordingly. Experiments on a number of challenging low-light
images are present to reveal the efficacy of our LIME and show its superiority over several state-of-the-arts in terms of enhancement quality and efficiency.
\section{Introduction}
 high-visibility images reflect clear details of target scenes, which are critical to many visionbased techniques, such as object detection and tracking. But, images captured in low-light conditions are often of low visibility. The visual quality of images captured under low-light conditions, for one thing, is barely satisfactory. For another thing, it very likely hurts the performance of algorithms that are primarily designed for high-visibility inputs. Directly amplifying the low-light image is probably the most intuitive and simplest way to recall the visibility of dark regions. But, this operation gives birth to another problem, say relatively bright regions might be saturated and thus loss corresponding details. Histogram equalization (HE) strategies [3]–[5] can avoid the above problem by somehow forcing the output image to fall in the range [0, 1]. Further, variational methods aim to improve the HE performance by imposing different regularization terms on the histogram. For instance, contextual and variational contrast enhancement (CVC) [6] tries to find a histogram mapping that pays
attention on large gray-level differences, while the work [7] achieves improvement by seeking a layered difference representation of 2D histograms (LDR). However, in nature, they focus on contrast enhancement instead of exploiting real illumination causes, having the risk of over- and underenhancement. Another solution is Gamma correction that is a nonlinear operation on images. The main drawback is that the nonlinear operation of Gamma correction is carried out on each pixel individually without considering the relationship of a certain pixel with its neighbors, and thus may make enhanced results vulnerable and visually inconsistent with real scenes.

In Retinex theory, the dominant assumption is that the (color) image can be decomposed into two factors,
say reflectance and illumination. Early attempts based on Retinex, such as single-scale Retinex (SSR) [9] and multi-scale Retinex (MSR), treat the reflectance as the final enhanced result, which often looks unnatural and frequently appears to be over-enhanced. The method proposed in [11] tries to enhance contrast while preserving naturalness of illumination. Although it prevents results from over-enhancement, in our experiments, it performs less impressive than our method in terms of both efficiency and visual quality. Fu et al. proposed a method to adjust the illumination by fusing multiple derivations of the initially estimated illumination map (MF) [12]. The performance of MF is mostly promising. But, due to the blindness of illumination structure, MF may lose the realism of regions with rich textures. The most recent work of [13] proposed a weighted variational model for simultaneous reflectance and illumination estimation (SRIE). With the estimated reflectance and illumination, the target image can be enhanced by manipulating the illumination. As noticed in [14], inverted low-light images look like haze images, as shown in Fig. 2. Based on this observation, the authors of [14] alternatively resorted to dehaze the inverted low-light images. After dehazing, the obtained unrealistic images are inverted again as the final enhanced results.


\section{Problem Definition}
Our system will enhance the input image by applying retinex algorithms like Single-Scale Retinex (SSR), Multi-Scale Retinex (MSR) and Multi-Scale Retinex with color Restoration (MSRCR) and power law transformation,Histogram equalization and Adaptive Histogram Equalization. Thus, the Problem definition can be proposed as follows: To implement the Algorithms: Retinex algorithma, power law transformation,Histogram equalization and Adaptive Histogram Equalization.
\subsection{Objective}
	\begin{itemize}
		\item To implement different Retinex Algorithms.
		\item To implement Power Law Transformation.
		\item To implement Histogram Equalization.
		\item To implement Adaptive Histogram Equalization
	\end{itemize}

\subsection{Contribution}
Our method belongs to the Retinex-based category, which intends to enhance a low-light image by
estimating its illumination map. It is worth noting that, different from the traditional Retinex-based methods like that decompose an image into the reflectance and the illumination components, our method only estimates one factor, say the illumination, which shrinks the solution space and reduces the
computational cost to reach the desired result. The illumination map is first constructed by finding the maximum intensity of each pixel in R, G and B channels. Then, we exploit the structure of the illumination to refine the illumination map.Also we have implemented and compare power law transformation,Histogram equalization and Adaptive Histogram equalization.

\section{Scope}
The Retinex Image Enhancement Algorithm is an automatic image enhancement method that enhances a digital image in terms of dynamic range compression, colour independence from the spectral distribution of the scene illuminate, and colour/ lightness rendition. Also implement and compare power law transformation,Histogram equalization and Adaptive Histogram Equalization with retinex algorithm. So, the scope of this project is to implement the Retinex and power law transformation,Histogram equalization and Adaptive Histogram Equalization based image Enhancement Algorithms, which involves the implementation of following three models of retinex algorithms and other:
	\begin{itemize}
		\item Multi-Scale Retinex algorithm with color Restoration (MSRCR)
		\item Multi-Scale Retinex algorithm with Chromaticity Preservation (MSRCP)
		\item Automated Multi-Scale Retinex algorithm with Color Restoration (AMSRCR)
		\item Power Law Transformation
		\item Histogram Equalization
		\item Adaptive Histogram Equalization

	\end{itemize}
\section{Outline}
This project report is organized into six chapters: Chapter 1 introduces the project; chapter 2 describes related work. Chapter 3 presents the proposed technology that will be used. Chapter 4 presents implementation, design and analysis of the system analysis.chapter 5 presents the results and discussion and last chapter 6 is the Conclusion.